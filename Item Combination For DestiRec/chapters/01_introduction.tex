% !TeX root = ../main.tex
% Add the above to each chapter to make compiling the PDF easier in some editors.

%The introduction shall provide the reader with an entryway to your topic. Commonly, the introduction is not too technical and provides the reader with a very general introduction why the topic of the thesis is relevant. Empirical examples are particular popular in introductions (make sure to provide citations)The introduction should also provide at least one research question that you try to answer. Last but not least, the introduction should also introduce the structure of the thesis (i.e., which chapters the reader should expect) in one paragraph.

\chapter{Introduction}\label{chapter:introduction}
\Glspl{rs} are systems that can produce individualized recommendations based on aggregating user's preferences \parencite{Ricci2011IntroductionHandbook}. They have the effect of guiding the user in a personalized way to interesting
or useful items in a large space of possible options \parencite{Burke2002HybridInteraction}. Given an input of the user's preferences, recommendation algorithms can generate a sequence of the recommended item(s).  \Glspl{rs} usually
apply their methodology in; Information Retrieval (IR), Human-Computer Interaction (HCI), and Data Mining (DM), fields \parencite{Ricci2011IntroductionHandbook}. Amazon.com \parencite{Linden2003Amazon.comFiltering} uses recommendation algorithms to personalize the online store for each customer, for example showing programming titles to a software engineer and baby toys to a new mother.  Netflix \parencite{Amatriain2013BigRecommendations} also use sophisticated recommendation algorithms to suggest movies to users based on user profile and behavior. \Glspl{rs} are also used to suggest articles, people, music, news e.t.c, in other e-commerce websites such as Spotify, LinkedIn, and others.

Although there are multiple domain types that \glspl{rs} can be applied to, a \gls{rs} typically focuses on a specific item type or domain. \Glspl{rs} also find valuable use cases in tourism. This thesis focuses solely on \glspl{rs} for the tourism domain. Travel recommendation and tourist trip planning is a popularly researched area with different proposed systems \parencite{wolfgang_umap_recsystem, cbrecsys2014, Thiengburanathum2018AnTourists, Arif2020Blockchain-BasedSystem, Alrasheed2020ASystem} and commercial destination recommendation tools (e.g. Triparoti\footnote{Triporati: \url{www.triporati.com}}, Tripzard\footnote{Tripzard: \url{www.tripzard.com}}, Besttripchoices\footnote{Besttripchoices: \url{www.besttripchoices.com}}).
Designing travel recommender systems is intractable because of the number of possible destinations and the complex
and multi-layered nature of the preferences and needs of tourists. Computing an optimal combination of destinations is an even more complex task. 

Travel recommendation can be described as a \gls{ttdp} \parencite{Vansteenwegen2007TheOpportunity}. A \gls{ttdp} model typically consists of a set of candidate \glspl{poi} each associated with a number of attributes (e.g., activities, location, etc.), and a \textit{score} of each \gls{poi}, calculated as a weighted function of the objective and/or the subjective value of each \gls{poi}. The objective of solving the \gls{ttdp} is to maximize the collected score of each sequence of ordered visits to the \glspl{poi} while respecting user constraints related to travel cost and \gls{poi} attributes \parencite{Survey_TTDP_Guavalas}. In its most basic form, the \gls{ttdp} is equivalent to an \gls{op} \parencite{Vansteenwegen2007TheOpportunity}. The \gls{op}, similar to the \gls{ttdp}, seeks to maximize the total collected profit by visiting selected nodes (i.e \gls{poi}) under a given value \parencite{T.1984HeuristicOrienteering}. In this problem, not all available nodes can be visited due to the limited time budget. Thus, a standard \gls{op} can be interpreted as a combination between the \textit{Knapsack Problem} and the \gls{tsp} \parencite{OP_Solution_Gunawan}. 

A number of \gls{op} variants focus on optimizing for best routes between \glspl{poi} (i.e., route planning) under \gls{poi} attributes constraints without or within a time frame. Hence, \glspl{op} can be viewed as the \gls{tsp} with profits. The \gls{ttdp} presented in this work does not include optimizing for routes between \glspl{poi} (i.e., we do not recommend order of visits to the chosen destinations). The objective function of an \gls{op} is typically modeled similar to a knapsack problem model. In its basic form, the goal of a knapsack problem is to maximize the value of the items placed in a knapsack without going over a weight limit or capacity. Intuitively, the sequence of trips to be recommended by our recommender system can be described as the knapsack, while the items to be placed in the knapsack are the single \glspl{poi} (i.e., destination). The weight limit enforced by the knapsack can be thought of as the budget and time constraint, while the value to be maximized by the knapsack problem is the score of each \gls{poi} for given user preferences and \gls{poi} attributes.

To illustrate the challenge of designing recommender systems for composite trips, consider the following scenario: A person wants to travel for a 3 weeks holiday in March, and she has a budget of €1500. Depending on her traveling style, her preferred activities may include going hiking, biking and/or visiting cultural attractions. The recommender system has to recommend a combination of destinations from many possible destinations while respecting her time and budget limitations. The traveler would derive more satisfaction if the recommender system suggests a stay duration for each recommended destination. Furthermore, the recommender system must consider her preferred activities to avoid missing out on any of her preferred activities during the total trip. Additionally, for maximum value, the recommender system should consider factors like the weather of the chosen time of the year, security of the regions, and proximity of the destinations per composite trip sequence of recommendation. Due to the combinatorial explosion of possible number of solutions, enumerating exhaustively all possible solutions is impractical for moderate and large instances

In this thesis, we shall be investigating various state-of-the-art algorithms that have been used in research to solve the \gls{op} in \glspl{ttdp}. Solutions to real-life optimization problems usually have to be evaluated considering different points of view corresponding to multiple objectives, often in conflict.  We describe an \gls{op} as a \gls{moop}, where there are several pleasure categories for each \gls{poi} (e.g., shopping, cultural) with each \gls{poi} having a distinct profit per category. At the time of this thesis, the data that we have collated is not voluminous enough for us to consider deep learning techniques that could be applied to the \gls{ttdp}. Hence, we will not be researching on deep learning algorithms or techniques to use.

The main contributions of this thesis are as follows.

\begin{itemize}
    \item A formal definition of the composite trip recommendation problem as an optimization problem modeled as a multi-objective orienteering problem \gls{moop}
    \item Empirical research into how current state-of-the-art algorithms used for solving Orienteering Problems \glspl{op} and general multi-objective optimization problems can be extended to the \gls{moop}
    \item \ldots (\todo{name the algorithms}) to obtain a sequence of candidate solutions that satisfy user constraints in a destination recommendation system
    \item A working and scalable prototype that implements clearly defined extensions of algorithms used in the \gls{ttdp} domain
    \item Comparative study of different implemented algorithm's performance via metrics such as computational speed and user's satisfaction \todo{decide how to evaluate}
\end{itemize}


Subsequent parts of this thesis are structured as follows. In section \ref{chapter:literature_review}, we shall review the available literature on various state-of-the-art approaches to travel recommendation. In section \ref{chapter:analysis} we formally define the composite trip recommendation problem as a \gls{moop} and provide empirical research into possible algorithmic approaches for solving \gls{ttdp} and how they can be adapted to the \gls{moop}. In section \ref{chapter:prototype_implementation}, we describe our algorithms to identify candidate solutions to the \gls{moop}, and describe the implementation. We evaluate the results of our implementation in section \ref{chapter:evaluation}. Finally, in section \ref{chapter:conclusion} we discuss the results and propose possible paths for future research. 




