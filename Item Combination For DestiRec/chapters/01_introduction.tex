% !TeX root = ../main.tex
% Add the above to each chapter to make compiling the PDF easier in some editors.

%The introduction shall provide the reader with an entryway to your topic. Commonly, the introduction is not too technical and provides the reader with a very general introduction why the topic of the thesis is relevant. Empirical examples are particular popular in introductions (make sure to provide citations)The introduction should also provide at least one research question that you try to answer. Last but not least, the introduction should also introduce the structure of the thesis (i.e., which chapters the reader should expect) in one paragraph.

\chapter{Introduction}\label{chapter:introduction}
\Glspl{rs} can be described as systems that produce individualized recommendations
as output or has the effect of guiding the user in a personalized way to interesting
or useful items in a large space of possible options \parencite{Burke2002HybridInteraction}. Given an input of users preferences, recommendation algorithms generates one or more recommended item(s). A \gls{rs} typically focus on a specific item type. Amazon.com \parencite{} uses recommendation algorithms to personalize the online store for each customer, for example showing programming titles to a software engineer and baby toys to a new mother.  Netflix \parencite{Amatriain2013BigRecommendations} also use sophisticated recommendation algorithms to suggest movies to users based on user profile and behavior. These real world examples intuitively suggests that recommendation algorithms can find valuable use cases in tourism. Tourist destination recommendation and tour planning system is a heavily researched area with different proposed systems \parencite{wolfgang_umap_recsystem, cbrecsys2014, Thiengburanathum2018AnTourists, Arif2020Blockchain-BasedSystem, Alrasheed2020ASystem} and commercial destination recommendation tools \parencite{triparoti, Www.tripzard.com, Www.besttripchoices.com}.

Designing tourist destination recommender systems is a difficult task because of the amount of possible destinations as well as the complex
and multi-layered nature of the preferences and needs of tourists. Recommending a combination of destinations is an even more complex task. Consider the scenario where a person wants to travel for a 3 weeks in March, and she has a budget of 1500€; her preferred activities include going to the beach, shopping and visiting museums. The recommender system has to recommend a combination of destinations from a large space of possible destinations while respecting her time and budget limitations. Furthermore, the recommender system should ideally recommend how long she has to stay in each destination, and her preferences should be fully taken into account and covered, to obtain her full satisfaction. Additionally, the recommender system might take factors like destination proximity, weather and security into account for optimal results. This example illustrates the complexity of designing a system that should recommend composite destinations.

We can distinguish \Glspl{rs}  based on the issues they focus on and the techniques they use.
\textit{Content-based} recommendation systems try to recommend items similar to those a given user has liked in the past, whereas systems
designed according to the \textit{collaborative} recommendation approach identify users
whose preferences are similar to those of the given user and recommend items they
have liked \parencite{Balabanovic1997Content-BasedRecommendation}. \textit{Hybrid approaches} \parencite{Adomavicius2005TowardExtensions} combine collaborative and content-based methods. \textit{Knowledge-based} systems \parencite{Burke2000Knowledge-basedSystems} depend on knowledge models of the object domain for effective item recommendation. They are based on the needs and preferences that the user provides. Depending on the recommendation technique employed, destination recommendation systems can be classified as content-based, collaborative or knowledge based. 

\todo{Write about destination RS is an algorithmic optimization problem, and discuss how it relates to TTDP (not in detail)}

\todo{Discuss what will be done in subsequent sections. The paper 18-MCKP is a good inspiration poin}

