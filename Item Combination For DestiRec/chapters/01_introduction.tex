% !TeX root = ../main.tex
% Add the above to each chapter to make compiling the PDF easier in some editors.



\chapter{Introduction}\label{chapter:introduction}
\Glspl{rs} produce individualized recommendations based on aggregating users' preferences \parencite{Ricci2011IntroductionHandbook}. They have the effect of guiding the user in a personalized way to interesting
or valuable items in an ample space of possible options \parencite{Burke2002HybridInteraction}. Given an input of the user’s preferences, recommendation algorithms can generate a sequence of the recommended item(s). \Glspl{rs} are typically applied information retrieval (IR), human-computer interaction (HCI), and data mining (DM) fields \parencite{Ricci2011IntroductionHandbook}. Amazon \parencite{Linden2003Amazon.comFiltering} uses recommendation algorithms to personalize the online store for each customer, for example showing programming titles to a software engineer and baby toys to a new mother. Netflix \parencite{Amatriain2013BigRecommendations} also uses sophisticated recommendation algorithms to suggest movies based on user profiles and behavior. \Glspl{rs} are also used to suggest articles, people, music, news, etc., in e-commerce websites such as Spotify, LinkedIn, and others.

Although there are multiple domain types which \glspl{rs} can be applied, a \glspl{rs} typically focuses on a specific item type or domain. There are also valuable use cases of \glspl{rs} in tourism. This thesis focuses solely on \glspl{rs} in the tourism domain. Travel recommendation and trip planning are popularly researched areas with different proposed systems \parencite{wolfgang_umap_recsystem, cbrecsys2014, Thiengburanathum2018AnTourists, Arif2020Blockchain-BasedSystem, Alrasheed2020ASystem} and commercial destination recommendation tools (e.g. Triparoti\footnote{Triporati: \url{www.triporati.com}}, Tripzard\footnote{Tripzard: \url{www.tripzard.com}}, Besttripchoices\footnote{Besttripchoices: \url{www.besttripchoices.com}}).
Designing travel \glspl{rs} is intractable because of the number of possible destinations and the complex
and multi-layered nature of the preferences and needs of tourists. Computing an optimal combination of destinations is an even more complex task. 

Travel recommendation can be described as a \gls{ttdp} \parencite{Vansteenwegen2007TheOpportunity}. A \gls{ttdp} model typically consists of a set of candidate \glspl{poi}, each associated with a number of attributes (e.g., activities, location), and a score for each \gls{poi}, is calculated as a weighted function of the objective or subjective value of each \gls{poi}. The objective of solving the \gls{ttdp} is to maximize the collected score of each sequence of ordered visits to the \glspl{poi} while respecting user constraints related to travel cost and \gls{poi} attributes \parencite{Survey_TTDP_Guavalas}. In its most basic form, the \gls{ttdp} is equivalent to an \gls{op} \parencite{Vansteenwegen2007TheOpportunity}. The \gls{op}, similar to the \gls{ttdp}, seeks to maximize the total collected profit by visiting selected nodes (i.e., \glspl{poi}) of a given value \parencite{T.1984HeuristicOrienteering}. In this problem, not all available nodes can be visited due to the limited time budget. Thus, a standard \gls{op} can be interpreted as a combination of the Knapsack Problem and the \gls{tsp} \parencite{OP_Solution_Gunawan}. 

A number of \gls{op} variants focus on optimizing to find the best routes between \glspl{poi} (i.e., route planning) under \gls{poi} attribute constraints without or within a given time frame. Hence, \glspl{op} can be viewed as the \gls{tsp} with profits. The \gls{ttdp} presented in this work does not include optimizing for routes between \glspl{poi} (i.e., we do not recommend the order of visits to the chosen destinations). The objective function of an \gls{op} is typically modeled similarly to a knapsack problem. A standard knapsack problem is a 0–1 integer programming model and is formally defined as follows:
%\intertext{subject to}
\begin{align}\tag{1}
    maximize \qquad &\sum_{i=1}^n p_ix_i\label{eq:1a}\\
   \tag{2} subject \hspace{0.1cm} to \qquad &\sum_{i=1}^n w_ix_i \leq W\label{eq:1b} \\
    \tag{3}x_i \in \{0,1\}, \qquad &\forall \hspace{0.1cm} 1 \leq i \leq n\label{eq:1c}
\end{align}
where each item $i$ is associated with a profit $p_i$. The decision variable is, $x_i = 1$, when an item is placed inside the knapsack; otherwise it is $x_i=0$. The objective function \ref{eq:1a} is to maximize the total profit from collected items. Constraint \ref{eq:1b} limits the weight $w_i$ of items placed inside the knapsack based on the total capacity of the knapsack $W$. The goal of a knapsack problem is to maximize the value of the items placed in the knapsack without going over a weight limit or capacity. Intuitively, the sequence of trips to be recommended by our \gls{rs} can be described as the knapsack, while the items to be placed in the knapsack are the single \glspl{poi} (i.e., destinations). The weight limit enforced by the knapsack can be thought of as the budget and time constraints, while the value to be maximized by the knapsack problem is the score of each \gls{poi} for given user preferences and \gls{poi} attributes.

To illustrate the challenge of designing \glspl{rs} for composite trips, consider the following scenario: A person wants to travel for a three weeks’ holiday in March, and she has a budget of €1,500. Depending on her traveling style, her preferred activities may include hiking, biking, or visiting cultural attractions. The \gls{rs} must recommend a combination of destinations from many possible destinations while respecting the traveler’s time and budget limitations. The traveler will derive more satisfaction if the \gls{rs} suggests a stay duration for each recommended destination. Furthermore, the \gls{rs} must consider the user’s preferred activities so that they do not miss out on activities during their trip. Additionally, for maximum value, the \gls{rs} should consider factors like the weather during the chosen time of the year, security of the regions, and proximity of the destinations per the trip agenda. Exhaustively enumerating all possible solutions is impractical in moderate and large instances because of the combinatorial explosion of the number of possible solutions.

In this thesis, we investigate various state-of-the-art algorithms that have been used in research to solve the \gls{op} in \glspl{ttdp}. Solutions to real-life optimization problems usually must be evaluated considering different points of view corresponding to multiple objectives that are often in conflict. We describe an \gls{op} as a \gls{moop} in which there are several pleasure categories for each \gls{poi} (e.g., shopping, cultural) with each \gls{poi} having a distinct profit per category. At the time of this thesis, the data we collated is not voluminous enough for us to consider deep-learning techniques that could be applied to the \gls{ttdp}. Hence, we do not research deep-learning algorithms or techniques

The main goals of this thesis are as follows:

\begin{itemize}
    \item A formal definition of the composite trip recommendation problem as an optimization problem modeled as a \gls{moop};
    \item Empirical research into how current state-of-the-art algorithms used for solving \glspl{op} and general \glspl{moop} can be extended;
    \item An efficient algorithm to obtain a sequence of candidate solutions that satisfy user constraints in a destination \gls{rs};
    \item Varying adaptations and clearly defined implementations of the algorithm; and
    \item An online comparative study and an offline user evaluation of the performance of the implemented algorithm variants using metrics such as computational speed, user satisfaction, diversity, cohesiveness, and accuracy.
\end{itemize}


Subsequent parts of this thesis are structured as follows. In Chapter \ref{chapter:literature_review}, we review the available literature on various state-of-the-art approaches to travel recommendation. In Chapter \ref{chapter:analysis} we formally define the composite trip recommendation problem as a \gls{moop} and provide empirical research into possible algorithmic approaches for solving \gls{ttdp} and how they can be adapted to the \gls{moop}. In Chapter \ref{chapter:prototype_implementation}, we describe our algorithms to identify candidate solutions to the \gls{moop} as well as their implementation. We evaluate the results of our implementation in Chapter \ref{chapter:evaluation}. Finally, in Chapter \ref{chapter:conclusion} we discuss the results and propose possible paths for future research. 




