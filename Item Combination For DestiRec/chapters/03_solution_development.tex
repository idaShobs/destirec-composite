%Obviously, here you describe in great detail the actual analysis you conducted. The level of detail should be sufficient to allow a very smart fellow student in your field to reproduce more or less your research.
\chapter{Solution Development}\label{chapter:analysis}
The framework for our recommender system comprises a knapsack problem, a data model, and an algorithmic approach. In this section, we shall be analysing these three parts in great detail. Thereafter we shall formally define our knapsack problem.

\section{The Knapsack Model}
A standard knapsack problem is an 0-1 integer programming model and is generally formally defined as follows.
\begin{gather}
    maximize \qquad \sum_{i=1}^n p_ix_i\label{eq:1a}
    \intertext{subject to}
    \sum_{i=1}^n w_ix_i \leq W\label{eq:1b} \\
    x_i \in \{0,1\}, \qquad \forall 1 \leq i \leq n\label{eq:1c}
\end{gather}
where each item $i$ is associated with a profit $p_i$. The decision variable $x_i = 1$ if an item is placed inside the knapsack, otherwise $x_i=0$. The objective function \ref{eq:1a} is to maximize the total profit from collected items. Constraint \ref{eq:1b} limits the weight $w_i$ of items placed inside knapsack by the total capacity of the knapsack $W$.

There are several variations of the knapsack problem. If the set of item is partitioned into classes, then the problem becomes multiple choice knapsack problem \gls{mckp} which is a generalization of the knapsack model. The binary choice of taking an item is replaced by the selection of exactly one item out of each class of items\parencite{Kellerer2004TheProblem}. Thus, given a partition of $n$ items, defined by $\{N_1, N_2,.., N_n\}$, where $N_i$ is represents an item class, the objective function in the \gls{mckp} is formally defined by
\begin{align}
    maximize \qquad \sum_{i=1}^n \sum_{j \in N_i} p_{ij}x_{ij}\label{eq:1d}
\end{align}
where $x_{ij}$ denotes the binary selection of an item from a class. The constraint \ref{eq:1b} is replaced and extended by 
\begin{gather}
    \sum_{i=1}^n \sum_{j \in N_i} w_{ij}x_{ij} \leq W,\label{eq:1e} \\
    \sum_{j \in N_i} x_{ij} = 1, \qquad \forall 1 \leq i \leq n\label{eq:1f}
\end{gather}
Constraint \ref{eq:1f} denotes that only one item of the class $j$ can be added into the knapsack. When identical copies of an item exists a class can be placed inside the knapsack, then it becomes a \textit{bounded knapsack problem}, where each item $x_i$ is bounded by $b_i$\parencite{Kellerer2004TheProblem}. Therefore, $x_i \in \{0,1\}$ in \ref{eq:1c} becomes $0\leq x_i \leq b_i$. The variable $b_i$ represents the number of identical copies. An \textit{unbounded knapsack problems} is defined when there are infinite identical copies (i.e., $b_i = +\infty$. However, unbounded knapsacks are not relevant for formulating our knapsack model.

\subsection{Types of Knapsack problem}

\section{Algorithmic approaches and Heuristics}
\subsection{Dynamic Programming}
\subsection{Constraint Programming}
\subsection{Branch and Bound}
\subsection{Column Generation}
\subsection{Genetic Algorithms}
\subsection{Ant Colony Optimization}


\subsection{Meta-Heuristics}
\subsubsection{Tabu Search}
\subsubsection{Simulated Annealing}
\subsubsection{Iterated Local Search}
\subsubsection{Variable Neighborhood Search}

\section{Data Model}

\section{Problem Formulation}
\todo{Mathematical formulation of problem as a Multiple Choice Knapsack Problem}
\todo{Oregon Trail Knapsack problem formulation}




