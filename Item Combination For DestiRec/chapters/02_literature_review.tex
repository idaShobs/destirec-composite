%The literature review is a core element of your thesis and shows that you are capable of working scientifically. As you explain what other researchers have found on your topic, the reader will realize that you know this topic extremely well. This will build trust that you can provide a piece of work yourself that is scientifically relevant. Equally important, you will need to identify a gap in the literature that you intent to fill. This is how you justify your thesis, and it helps the reader to assess the importance of your work. This gap may be methodological ("I will develop a new method that is able to answer my research question, which previously applied methods cannot as well."), use new data ("Other researchers used database X, but I will use data retrieved by Y."), or a new application ("This method has never been applied to the city of Munich.").
%Summarize and synthesize: give an overview of the main points of each source and combine them into a coherent whole
%Analyze and interpret: don’t just paraphrase other researchers—add your own interpretations where possible, discussing the significance of findings in relation to the literature as a whole
%Critically evaluate: mention the strengths and weaknesses of your sources
%Write in well-structured paragraphs: use transition words and topic sentences to draw connections, comparisons and contrasts

\chapter{Literature Review}\label{chapter:literature_review}
In 2007, Vansteenwegen \& Oudheusden\parencite{Vansteenwegen2007TheOpportunity} published a popular literature for \gls{ttdp} design. Their paper defined features that could be described as a template for the next generation \gls{mtg}. Through their research, the authors projected that the next generation \gls{mtg} would become more personalized with components such as user profile, attraction information, and trip information playing essential roles. They also speculated that \gls{or} would play a pivotal role in conceiving and optimizing tourist trips. Thus, they defined the \gls{ttdp} as a decision tool.

Since their groundbreaking projections and proposals, the evolution of technologies such as \gls{ai}, \gls{iot} and \gls{cps}, have caused a paradigm shift in the possibilities for a dynamic tourist destination guide. Many recent studies have focused on designing and optimizing tourist trips via the aid of technological advancements.  Existing literature has thoroughly researched \Gls{ttdp} as a decision tool for the new generation \gls{mtg} has thoroughly. However, it is has become recently less mainstream because of the options offered by newer technologies. It has become the de facto approach to personalize new generation tourist guides or trip recommendation systems. We shall explore such personalized recommender systems after surveying recommendation techniques commonly found in literature.

\section{Recommendation Techniques}
Generally, we can distinguish \Glspl{rs}  based on the issues they focus on and the techniques they use. As shown in figure \ref{fig:recSystems-approaches}, recommender systems can be based on collaborative filtering (also known as social filtering), content-filtering, and knowledge-based filtering.
Systems designed according to \textit{collaborative filtering} identify users whose preferences are similar to those of the given user by computing similarities between users and recommending items they have liked\parencite{Balabanovic1997Content-BasedRecommendation}. Recommender systems extensively use this filtering method. Collaborative filtering often comprise model-based and memory-based approaches. Model-based approaches typically use machine learning or statistical methods to build models based on user records for future recommendations \parencite{Chu2012DoesImages}. In contrast, memory-based approaches compare user data with previously stored data of other users\parencite{Schiaffino2006PoliteAgents}. This technique typically suffers from a \textit{cold-start} problem. Such a problem occurs when there is not enough information about an item through another user 
either rating it or specifying which other items it is similar to \parencite{Balabanovic1997Content-BasedRecommendation}. Also, data becomes sparse when there are not enough users available to cover the needed collection of recommendable items.
\parencite{Gao2015Content-awareNetworks, Lian2015Content-AwareData} analyze a set of items liked by a user in the past and try to recommend similar items, resulting in a judgment of relevance that represents the users' preferences. Thus, it is highly advantageous for the effectiveness of an information retrieval process. Content-filtering-based systems suffer from over-specialization. This is a function of the similarity between the recommended items based on users' preference at different points \parencite{Lops2011Content-basedTrends}. It also suffers from a cold-start problem, whereby the system needs to have adequate user historical records to generate quality results \parencite{Burke2002HybridInteraction}. 

\textit{Knowledge-based} systems \parencite{Burke2000Knowledge-basedSystems} depend on knowledge models of the object domain for effective item recommendation. They are based on the needs and preferences that the user provides. Constraint-based techniques \parencite{Choi2021ATourismcity} allow systems to use domain knowledge about the features that cannot be represented by the user's record or in the absence of the user's record to recommend items. Case-based methods \parencite{Montejo-Raez2011Otium:Leisure} typically utilize the experiences and expertise of travel agents. Knowledge-based methods do not suffer from the problems faced in collaborative and content-filtering. Its recommendations do not depend on a base of user ratings, and it does not have to gather information about a particular user because its
judgments are independent of individual tastes  \parencite{Burke2000Knowledge-basedSystems}. Therefore, they are perfect complements to other recommender systems types. 

\textit{Hybrid approaches} \parencite{Adomavicius2005TowardExtensions,Ghazanfar2010BuildingFiltering} combine two or more of the approaches mentioned above to mitigate the weaknesses of each approach and complement their strengths for optimal performance. 

Depending on the recommendation technique employed, destination recommendation systems can be classified as content-based, collaborative, or knowledge-based. However,  a good deal of recent research has heavily focused on model-based collaborative filtering.


\begin{figure}[htpb]
  \centering
   \documentclass{standalone}
\begin{document}
\tikzset{
  basic/.style  = {draw, text width=2cm, drop shadow, font=\sffamily, rectangle},
  root/.style   = {basic, rounded corners=2pt, thin, align=center, fill=TUMAccentBlue, inner sep=6pt},
  level 2/.style = {basic, rounded corners=6pt, thin,align=center, fill=TUMAccentBlue, text width=8em, align=center,  inner sep=6pt},
  level 3/.style = {basic, thin, align=left,  text width=6.5em, align=center,  inner sep=4pt}
}
 \begin{tikzpicture}[
    level 1/.style={sibling distance=5cm},
    level 2/.append style={sibling distance=35mm},
   level distance = 2cm,  edge from parent fork down]
    
    \node [root]{Recommender Systems}
        child {node [level 2]{Collaborative \\ filtering}
        child {node [level 3] {Model-Based}}
        child {node [level 3] {Memory-Based}}
        }
        child {node [level 2] {Content-Based \\ filtering}}
        child {node [level 2] [sibling distance =2cm] {Knowledge-Based\\ filtering}
        child {node [level 3] {Constraint-Based}}
        child {node [level 3] {Case- \\Based}}
        };
    \end{tikzpicture}
\end{document}
    \caption[Recommender Systems Approaches Overview]{An overview of approaches used in recommender systems.}\label{fig:recSystems-approaches}
\end{figure}

\section{POI Recommendation}
The proliferation of social media as a societal norm has paved the way for \Glspl{lbn} like Foursquare and Facebook places. Consequently, this has pioneered establishing a recent study area in destination recommendation research. \gls{poi} recommendation, as it is commonly known, is an active research area that focuses solely on recommending destinations to users on \gls{lbn}. Such systems utilize readily available check-in records of a user on the network to recommend further possible \glspl{poi} to the user. Strides in \gls{poi} recommendation should be explored because they constitute a significant step towards extracting user preference data that can be generalized into improving destination recommendation in systems that are also outside location-based networks.

The main objective of \gls{poi} recommendation on \glspl{lbn} is to optimize the user's experience by recommending \glspl{poi} based on the user's historical record and sometimes collaborative information. \Glspl{lbn} are broadly interesting for research in recent years due to the spatial-temporal-social information embedded in them\parencite{Cheng2011ExploringServices}. Typically, users check-in data with timestamps, location details, and their preferences inform of \gls{poi} ratings, which can be accessed via an API provided by the \gls{lbn} provider. As a consequence thereof, many recent studies have experimented on representing the data sets through statistical or machine models to mine patterns in the \gls{lbn} data for the \gls{poi} prediction task. The data characteristics allow the definition of new properties that can be mined from them.

\Gls{lda} is a natural language processing statistical model that is heavily adopted by researchers to extract \gls{poi} topics from comment sections \parencite{Liao2018POIFactorization, Huang2020Multi-modalNetworks}. Liao et al.\parencite{Liao2018POIFactorization} explored comment relations using \gls{lda}. In their study, their latent features were user-topic-time \glspl{tensor}. The \glspl{tensor} were constructed by connecting check-in data with a \gls{poi} topic generated by a \gls{lda} model. The \gls{poi} model extracts the topics based on the comments given by users, the topics and \gls{poi}-topic distributions of all \glspl{poi}. 

Some of the most successful realizations of statistical models for recommendation systems are based on \textit{matrix factorization}\parencite{Cheng2013WhereRecommendation, Cheng2012FusedNetworks, Chen2018PrivacyFactorization, Gao2013ExploringNetworks, Lian2015Content-AwareData}. Matrix factorization\parencite{Koren2009MatrixSystems} algorithms work by mapping items to a joint latent factor space of dimensionality, such that user-item interactions are modeled as inner products in that space. Consequently, high correspondence between item and user factors leads to a recommendation. A major strength of matrix factorization is that it allows the incorporation of additional information. In \gls{lbn} the \textit{social relations} between users, has helped incorporate similarity information between users interest into matrix factorization\parencite{Cheng2013WhereRecommendation, Liao2018POIFactorization, Huang2020Multi-modalNetworks}. Liao et al.\parencite{Liao2018POIFactorization} used an \gls{hosvd} of third-order \glspl{tensor} to recommend \glspl{poi}. \Gls{hosvd} can be considered as a generalization of a matrix singular value decomposition \parencite{Vandewalle1990SingularProcessing}. However, factorizing the matrix often raises difficulties due to the high portion of missing values caused by sparseness in the matrix due to cold-start. Huang et al.\parencite{Huang2020Multi-modalNetworks} combated this by simultaneously mining the sequential, temporal, and spatial patterns of users check-in behavior such that data are abundant. Others like Cheng et al.\parencite{Cheng2013WhereRecommendation} have employed a low-rank approximation of sparse partially observed \glspl{tensor}.

Apart from the social relations that can be incorporated into matrix factorization or topics, many existing studies have used the Markov-chain property of inter check-in behavior to model the \textit{sequential check-in pattern} of users \parencite{Cheng2013WhereRecommendation, Zhao2016STELLAR:Recommendation, Yang2013ASystem}. Cheng et al. \parencite{Cheng2013WhereRecommendation} explored the \textit{geographical relation} of \gls{lbn} inter check-in distance in order to recommend successive \glspl{poi} to users through the use of Factorized Personalized Markov Chain(FPMC). Liu et al. \parencite{Liu2016PredictingContexts} have employed Recurrent Neural Networks (RNN) to detect subsequent correlations of a check-in sequence. 

Generally, due to the abundance of \gls{lbn} data available to the used models, the results obtained in many studies into \gls{poi} recommendation are mostly an improvement to previous studies. Also,  applying model-based techniques and through the use of approximation algorithms has helped achieve a breakthrough to the cold-start problem that could occur in collaborative filtering. The steady improvements of \gls{poi} recommendation precision and accuracy are indeed groundbreaking and a significant step towards improving destination recommendation as a whole. 


\section{OP solving methodologies}
Though \gls{poi} recommendation is an attractive research area, much research has also been done towards destination recommendation outside \glspl{lbn}. The \Gls{op} which originates from operations research, is still regarded as a tool for modeling the \gls{ttdp} and, in extension, solving the problem of destination recommendation. According to \parencite{Vansteenwegen2011TheSurvey}, given as set of nodes $N = \{1,...,|N|\}$ where each node $i \in N$ is associated with a non-negative score $S_i$ and the nodes $1$ and $N$ are the start and end nodes respectively. The goal of the \gls{op} is to determine a path, limited by a given time budget $T_{max}$, that visits a subset of $N$ and maximizes the total collected score. Each collected score can be added, and each node can be visited at most once. Many researchers have solved different problems by formulating them as variants of the \gls{op}. Such variants include \gls{top}, \gls{tdop}, \gls{optw}, \gls{toptw}, \gls{opsp}, \gls{moop}, and more recently the \gls{thop} have emerged in literature. 

Gunawan et al.(2016) \parencite{OP_Solution_Gunawan} already published a paper surveying all recent variations, solution approaches, and applications. Although numerous papers have thoroughly studied some \gls{op} variants, for some other variants, it is almost impossible to find in the literature. The \gls{moop} is one of the closest \gls{op} variant to our objective. \Gls{moop} derives its name from multi-objective optimization in \gls{or}, where there are more than one objective function. A \gls{moop} assigns a set of scores $S_{ik}$ to each \gls{poi} to model the different scores a \gls{poi} might have for each attraction category $k \in \{1,...,m\}$. Exact algorithms for the \gls{op} are complex and computationally time-consuming because it is NP-hard \parencite{Golden1987TheProblem}. Therefore, heuristic approaches such as the ones found in \parencite{T.1984HeuristicOrienteering, Golden1987TheProblem, Ramesh1991AnProblem,Souffriau2008AGuides} are the focus of most research. 

A good deal of literature have commonly used several meta-heuristics to solve different \gls{op} variants. Most of the meta-heuristics mimic natural metaphors to solve optimization problems (e.g., evolution of species, annealing process, ant colony, particle swarm, immune system, bee colony, and wasp swarm). Swarm algorithms such as \Gls{aoc}  \parencite{Ke2008AntsProblem, Wagner2016StealingProblem, Mukhina2019OrienteeringConstruction, Martin-Moreno2018Multi-ObjectiveProblem} and \gls{pso} \parencite{Sevkli2010StPSO:Optimization, Wagner2016StealingProblem, Schilde2009MetaheuristicsProblem} are meta-heuristic that have commonly found in literature for solving bi-objective and multi-objective \gls{op}. Vansteenwegen \& Oudheusden \parencite{Vansteenwegen2011TheSurvey}  in their survey listed \gls{aoc} based algorithms as one of the best-performing algorithms for the \gls{top} in their experiments. Memetic algorithms as used in \parencite{Lu2018AConstraints,Bouly2010AProblem, Divsalar2014ASelection} have also solved different variants of the \gls{op} problems. Evolutionary algorithms that mimic genetics are also an algorithmic trend in literature \parencite{Wu2018EvolutionaryProblem, Faeda2020AProblem, Kobeaga2018AnProblem}. Swarm algorithms, evolutionary algorithms, memetic algorithms, and simulated annealing as used in \parencite{Pan2018IndependentTourist}, are commonly combined with other heuristics such as tabu search \parencite{Lu2018AConstraints} or pure local search \parencite{Divsalar2014ASelection, Bouly2010AProblem} to achieve the best results. For example, Labadie et al. \parencite{Labadie2011HybridizedWindows} developed a hybridized evolutionary local search algorithm for \gls{toptw} that showed an improvement against 150 best-known solutions, as shown in \parencite{OP_Solution_Gunawan}. Recent studies have focused on exploring the strengths of many of these state-of-the-art meta-heuristics.  However, standard heuristics such as the Branch-and-cut algorithm used in  \parencite{Dang2013AProblem} and tabu search used in  \parencite{E.2005AProblem} are known to show great results for the \gls{top}. 

\section{Related Work}
A good deal of literature focus on either travel recommendation as a topic or bundling items to a recommendation package. However, there is little research into bundling items for destination recommendation. A non-time dependent \gls{moop} best describes our \gls{op} variant. To the best of our knowledge, no literature from research has defined travel recommendation as this \gls{op} variant. Martín-Moreno \& Vega-Rodríguez \parencite{Martin-Moreno2018Multi-ObjectiveProblem} proposed a swarm intelligence-based algorithm to solve a time-dependent \gls{moop}. However, due to little work done in \gls{op} for multi-objectives, benchmark instances to compare the performance of their algorithm are bi-objective \gls{op} based. Moreover, the \gls{moop} presented seeks to find optimal tours and not paths (i.e., start at node A and end at node A).

At the core of this work, are aspects relating to other works found in literature. The next sections highlights two most related works.


\subsection{The Oregon Trail Knapsack} \label{sec:oregon}
Burg et al. \parencite{Oregon_Trail_Knapsack} defined an Oregon trail knapsack problem. The Oregon trail game inspired their idea. The game asks the players to imagine preparing for a trek across the Oregon Trail. With a given amount of money to spend and supplies weight limit, the travelers need to get good value for the supplies they purchase to make it across the country. The extension mentioned above of the knapsack problem can be considered similar to our knapsack problem. However, the Oregon trail knapsack problem does not add a time constraint to the travel node as found in standard \gls{op}. Their formal definition of the problem was as follows:
%\intertext{subject to}
\begin{align}\tag{1}
    maximize \qquad &\sum_{j=1}^n f_j(x_j, x_{d_j})\label{eq:2a}\\
    \tag{2}subject \hspace{0.1cm} to \qquad &\sum_{j=1}^n w_j x_j \leq W\label{eq:2b} \\
  \tag{3} &\sum_{j=1}^n c_j x_j \leq C\label{eq:2c} \\
   \tag{4}  0\leq x_j \leq b_j, \qquad &\forall \hspace{0.1cm} 1 \leq j \leq n\label{eq:2d}
\end{align}
where $W$ in \ref{eq:2b} is the weight limit, and $C$ in \ref{eq:2c} is the cost limit. The function in \ref{eq:2a} is the value function for type $j$, in which $d_j$ gives the index of the type upon which the value of $x_j$ depends.  $d_j \in \{d_1, d_2,...,d_n\}$ provides the dependency information among the item type. The value function in this model is of particular interest because it models three possible value types of the items. Burg \& Lang\parencite{Oregon_Trail_Knapsack} defined the value functions as follows:

\noindent
\begin{flalign}
\tag{type 1 }\qquad f_j(x_j, x_{d_j}) &= x_j \cdot v_j \cdot [x_{d_j} > 0] \label{eq:2e}&&\\
\tag{type 2} \qquad f_j(x_j, x_{d_j}) &= [x_{d_j} > 0] \cdot \sum_{i=0}^{x_j - 1 } r^i \cdot v_j \label{eq:2f}&&\\&= [x_{d_j} > 0] \cdot v_j \cdot \frac{1-r^{x_j}}{1 - r}, \hspace{2.1cm} \hfill\forall \hspace{0.1cm} 0 < r < 1\notag &&\\
\tag{type 3} \qquad f_j(x_j, x_{d_j}) &= x_j \cdot v_j - [x_{d_j} > 0] \cdot t \cdot x_j \cdot v_j, \qquad  \hfill \forall \hspace{0.1cm} 0 < t \leq 1  \wedge d_j \neq j\label{eq:2g}
\end{flalign}

Each function has a value constant given by $v_j$. The Iverson notation $[x_{d_j} > 0]$ denotes the boolean value $x_{d_j} \in \{0,1\}$, which is when  $x_{d_j} > 0 $. Value function  \ref{eq:2e} represents the case, where items of type $j$ only have a value if  at least one item of type $d_j$ is present in the solution. The value function  \ref{eq:2f} covers the value type, where with each added item of type $j$, the item value $v_j$ diminish at a rate of $r$. Once more, the function adds the value of items of type $j$ if at least one item of type $d_j$ is in the knapsack. Lastly, the function \ref{eq:2g} represents the case where a factor of $t$ reduces the value associated with $x_j$ of type $j$ items, when type $x_{d_j}$ items are in the solution. 

\subsection{Composite Trip Recommendation}
Wörndl \& Herzog \parencite{cbrecsys2014} have previously done research on combining multiple travel regions into a composite trip. Their algorithmic approach comprises three phases:
\begin{enumerate}
    \item Reduce number of regions
    \item Rate regions
    \item Calculate the best combination of regions
\end{enumerate}

Phase one and two are preliminary steps taken before the actual recommendation process. Their approach allows users to exempt regions from their query. The hierarchical tree data model used, enables the procedure to exempt subregions of exempted regions. Thus, the amount of possible regions to explore is significantly reduced before starting the next phases. In phase two, an asymmetric similarity metric was used to assign preference ranks to regions. The asymmetrical similarity metric allows region features to be accorded weights. For example, the traveling period is weighted higher than features like crime level. Regions with low scores are then removed from consideration for the next step. Finally, the best combination of regions are then calculated with using dynamic programming.


Intrinsic details about the dynamic programming approach used in the study are unclear. The study was proven to satisfy expert users more than a baseline knapsack algorithm and a top-k algorithm. 


\section{Conclusion}
In this section, we have reviewed different techniques found in literature for item recommendation. We investigated state-of-art methodologies found in literature for general \gls{poi} recommendation and then went on to review how orienteering problems have been solved in literature.
Wörndl \& Herzog \parencite{cbrecsys2014} did great work on item combination for destination recommender systems. However, the algorithm fails to achieve high diversity in regions \parencite{cbrecsys2014} and their approach fails to scale. Penalty-based approach as used by their work can either improve a heuristic efficiently, or it could restrict the goodness of the heuristic. This depends on the penalty function definition and the chosen penalty value. This work seeks to improve their work by exploring methodologies and techniques to efficiently optimize the item combination procedure. Also, this thesis contributes to existing research on orienteering problems by characterizing a multi-objective non-time dependent orienteering problem. Hereby, more efficient heuristics for computing the combination of regions can be explored. Overall, higher diversity in recommended regions is to be ensured. A robust framework of problem model and scalable algorithmic approaches to building our recommender system for recommending diverse sequences of travel regions based on given user preference shall be developed in subsequent sections.
