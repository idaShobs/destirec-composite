%The literature review is a core element of your thesis and shows that you are capable of working scientifically. As you explain what other researchers have found on your topic, the reader will realize that you know this topic extremely well. This will build trust that you can provide a piece of work yourself that is scientifically relevant. Equally important, you will need to identify a gap in the literature that you intent to fill. This is how you justify your thesis, and it helps the reader to assess the importance of your work. This gap may be methodological ("I will develop a new method that is able to answer my research question, which previously applied methods cannot as well."), use new data ("Other researchers used database X, but I will use data retrieved by Y."), or a new application ("This method has never been applied to the city of Munich.").
%Summarize and synthesize: give an overview of the main points of each source and combine them into a coherent whole
%Analyze and interpret: don’t just paraphrase other researchers—add your own interpretations where possible, discussing the significance of findings in relation to the literature as a whole
%Critically evaluate: mention the strengths and weaknesses of your sources
%Write in well-structured paragraphs: use transition words and topic sentences to draw connections, comparisons and contrasts

\chapter{Literature Review}\label{chapter:literature_review}
In 2007, Vansteenwegen et al published their famous research \parencite{Vansteenwegen2007TheOpportunity} and defined features that could be described as a template for the next generation \gls{mtg}. The authors projected that the next generation \gls{mtg} will become more personalized with components such as user profile, attraction information and trip information playing important roles. They also speculated that \gls{or} will play a pivotal role in conceiving and optimizing tourist trips, thus, they defined the \gls{ttdp} as a decision tool.

Since their groundbreaking research, the evolution of technologies such as \gls{ai}, Internet-Of-Things and Cyber-Physical Systems, have caused a paradigm shift in the possibilities for a dynamic tourist destination guide. Many recent studies have focused on solving the problem of designing and optimizing tourist trips via the aid of technological advancements. \Gls{ttdp} as a decision tool for the new generation \gls{mtg} has thoroughly been researched in literature, however, it is has become recently less mainstream because of the options offered by newer technologies. It has become the de-facto approach to personalize new generation tourist guides or trip recommendation systems. We shall explore such personalized recommender systems after surveying recommendation techniques commonly found in literature.

\section{Recommendation Techniques}
Generally, we can distinguish \Glspl{rs}  based on the issues they focus on and the techniques they use. As shown in figure \ref{fig:recSystems-approaches}, recommender systems can be based on collaborative filtering (also known as social filtering), content-filtering and knowledge-based filtering.
Systems designed according to \textit{collaborative filtering} identify users whose preferences are similar to those of the given user by computing similarities between users and recommend items they have liked\parencite{Balabanovic1997Content-BasedRecommendation}. This filtering method is extensively used in recommender systems. Model-based and memory-based approaches are often categorized under collaborative filtering. Model-based approaches typically use machine learning or statistical methods to build models based on user records for future recommendation \parencite{Chu2012DoesImages}, whereas, memory-based approaches compare user data with previously stored data of other users\parencite{Schiaffino2006PoliteAgents}. This technique typically suffers from a \textit{cold-start} problem. Such a problem occurs when there is not enough information about an item through another user 
either rating it or specifying which other items it is similar to \parencite{Balabanovic1997Content-BasedRecommendation}. Also data becomes sparse when there isn't enough users available to cover the needed collection of recommendable items.

\textit{Content-based filtering} aided systems\parencite{LianContent-awareData, GaoContent-awareNetworks} analyze a set of items liked by a user in the past and try to recommend similar items. This results in a relevance judgement that represents the users preferences. Thus,it is highly advantageous for the effectiveness of an information retrieval process. Content-filtering based systems suffer from over-specialization. This is a function of the similarity between the recommended items based on users preference at different points \parencite{Lops2011Content-basedTrends}. It also suffers from a cold-start problem, whereby the system needs to have adequate users historical record in order to generate quality results \parencite{Burke2002HybridInteraction}. 

\textit{Knowledge-based} systems \parencite{Burke2000Knowledge-basedSystems} depend on knowledge models of the object domain for effective item recommendation. They are based on the needs and preferences that the user provides. Constraint-based techniques \parencite{Choi2021ATourismcity} allows systems use domain knowledge about the features that cannot be represented by user's record or in the absence of user's record to recommend items. Case-based methods\parencite{Montejo-Raez2011Otium:Leisure} typically utilize past-experiences and expertise of travel agents. Knowledge-based methods do not suffer from the problems faced in collaborative and content-filtering. Its recommendations do not depend on a base of user ratings and it does not have to gather information about a particular user because its
judgements are independent of individual tastes\parencite{Burke2000Knowledge-basedSystems}. Therefore, they are perfect complements to other recommender systems types. 

\textit{Hybrid approaches}\parencite{Adomavicius2005TowardExtensions,Ghazanfar2010BuildingFiltering} combine two or more of the aforementioned approaches in order to mitigate the weaknesses of each approach and complement their strengths for optimal performance. 

Depending on the recommendation technique employed, destination recommendation systems can be classified as content-based, collaborative or knowledge based. However, many recent research have heavily focused on model-based collaborative filtering.


\begin{figure}[htpb]
  \centering
   \documentclass{standalone}
\begin{document}
\tikzset{
  basic/.style  = {draw, text width=2cm, drop shadow, font=\sffamily, rectangle},
  root/.style   = {basic, rounded corners=2pt, thin, align=center, fill=TUMAccentBlue, inner sep=6pt},
  level 2/.style = {basic, rounded corners=6pt, thin,align=center, fill=TUMAccentBlue, text width=8em, align=center,  inner sep=6pt},
  level 3/.style = {basic, thin, align=left,  text width=6.5em, align=center,  inner sep=4pt}
}
 \begin{tikzpicture}[
    level 1/.style={sibling distance=5cm},
    level 2/.append style={sibling distance=35mm},
   level distance = 2cm,  edge from parent fork down]
    
    \node [root]{Recommender Systems}
        child {node [level 2]{Collaborative \\ filtering}
        child {node [level 3] {Model-Based}}
        child {node [level 3] {Memory-Based}}
        }
        child {node [level 2] {Content-Based \\ filtering}}
        child {node [level 2] [sibling distance =2cm] {Knowledge-Based\\ filtering}
        child {node [level 3] {Constraint-Based}}
        child {node [level 3] {Case- \\Based}}
        };
    \end{tikzpicture}
\end{document}
    \caption[Recommender Systems Approaches Overview]{An overview of approaches used in recommender systems.}\label{fig:recSystems-approaches}
\end{figure}

\section{POI Recommendation}
The proliferation of social media as a societal norm has paved the way for \Glspl{lbn} like Foursquare and Facebook places. Consequentially, this has pioneered the establishment of a recent study area in destination recommendation research. \gls{poi} recommendation, as it is commonly known, is an active research area that focuses solely on recommending destinations to users on \gls{lbn}. Such systems utilize readily available check-in records of a user on the network to recommend further possible \glspl{poi} to the user. Strides in \gls{poi} recommendation should be explored because they are a major step towards extracting user preference data that can be generalized into improving destination recommendation in systems that are also outside location based networks.

The main objective of \gls{poi} recommendation on \glspl{lbn} is to optimize users experience by recommending \glspl{poi} based on users historical record and sometimes collaborative information. \Glspl{lbn} are broadly interesting for research in recent years due to the spatial-temporal-social information embedded in them\parencite{Cheng2011ExploringServices}. Typically, users check-in data with timestamps, location details, and their preferences inform of \gls{poi} ratings, can be accessed via an API provided the \gls{lbn} provider. As a consequent thereof, many recent studies have experimented on representing the data sets through statistical or machine models in order to mine patterns in the \gls{lbn} data for the \gls{poi} prediction task. The characteristics of the data allows the definition of new properties that can be mined from them.

\Gls{lda} is a natural language processing statistical model that is heavily adopted by researchers to extract \gls{poi} topics from comment sections\parencite{Liao2018POIFactorization, Huang2020Multi-modalNetworks}. Liao et al also explored comment relations. Their latent features were user-topic-time tensors\footnote{Tensors can be seen as a generalization of matrices and are often used inter-changeably with matrices in literature.}. The tensors were constructed by connecting check-in data with a \gls{poi} topic generated by a \gls{lda} model. The \gls{poi} model extracts the topics based on the comments given by users, the topics and \gls{poi}-topic distributions of all \glspl{poi}. 

Some of the most successful realizations of statistical models for recommendation systems are based on \textit{matrix factorization}\parencite{Cheng2013WhereRecommendation,ChengFusedNetworks, Chen2018PrivacyFactorization, GaoExploringNetworks, LianContent-awareData}. Matrix factorization\parencite{Koren2009MatrixSystems} algorithms work by mapping both users and items to a joint latent factor space of dimensionality, such that user-item interactions are modeled as inner products in that space. Consequentially, high correspondence between item and user factors leads to a recommendation. A major strength of matrix factorization is that it allows the incorporation of additional information. In \gls{lbn} the \textit{social relations} between users has helped incorporate similarity information between users interest into matrix factorization\parencite{Cheng2013WhereRecommendation, Liao2018POIFactorization, Huang2020Multi-modalNetworks}. Liao et al.\parencite{Liao2018POIFactorization} used a \gls{hosvd} of third-order tensors to recommend \glspl{poi}. \Gls{hosvd} can be considered as a generalization of a matrix singular value decomposition\parencite{Vandewalle1990SingularProcessing}. However, factorizing the matrix often raises difficulties due to the high portion of missing values caused by sparseness in the matrix due to cold-start. Huang et al.\parencite{Huang2020Multi-modalNetworks} combated this by simultaneously mining the sequential, temporal and spatial patterns of users check-in behavior such that there is an abundance of data. Others like Cheng et al.\parencite{Cheng2013WhereRecommendation} have employed a low-rank approximation of sparse partially observed tensors.

Apart from the social relations that can be incorporated into matrix factorization or topics, many existing studies have used the Markov-chain property of inter check-in behavior to model the \textit{sequential check-in pattern} of users\parencite{Cheng2013WhereRecommendation, Zhao2016STELLAR:Recommendation, YangACoRR}. Cheng et al explored the \textit{geographical relation} of \gls{lbn} inter check-in distance in order to recommend successive \glspl{poi} to users through the use of Factorized Personalized Markov Chain(FPMC). Liu et al.\parencite{LiuPredictingContexts} have employed Recurrent Neural Networks (RNN) to detect subsequent correlations of a check-in sequence. 

Generally, the results obtained in a lot of studies into \gls{poi} recommendation are mostly an improvement to previous studies. This is due to abundance \gls{lbn} data available to the models been used. Also, major breakthrough to the cold-start problem that could occur in collaborative filtering has been achieved by applying model-based techniques and through the use of approximation algorithms. The steady improvements of \gls{poi} recommendation precision and accuracy are indeed groundbreaking and a major step towards improving destination recommendation as a whole. 


\section{OP solving methodologies}
Though \gls{poi} recommendation is an attractive research area, a lot of research has also been done towards destination recommendation outside \glspl{lbn}. The \Gls{op} which originates from operations research, is still regarded as a tool for modelling the \gls{ttdp} and in extension solving the destination recommendation problem. According to \parencite{Vansteenwegen2011TheSurvey}, given as set of nodes $N = {1,...,|N|}$ where each node $i \in N$ is associated with a non-negative score $S_i$ and the nodes $1$ and $N$ are the start and end nodes respectively. The goal of the \gls{op} is to determine a path, limited by a given time budget $T_{max}$, that visits a subset of $N$ and maximizes the total collected score. Each collected scores can be added and each node can be visited at most once. Many researchers have solved different problems by formulating them as variants of the \gls{op}. Such variants include \gls{top}, \gls{tdop}, \gls{optw}, \gls{toptw}, \gls{opsp} and more recently the \gls{thop} have emerged in literature. Gunawan et al.\parencite{OP_Solution_Gunawan} in 2016 published a paper surveying all recent variations, solution approaches and applications. While some \gls{op} variants have been thoroughly researched with a number of papers available about them, for some others little can be found in the literature. Exact algorithms for the \gls{op} are complex and computationally time-consuming because it is NP-hard\parencite{Golden1987TheProblem}, therefore most research are focused on heuristic approaches such as the ones found in \parencite{T.1984HeuristicOrienteering, Golden1987TheProblem, Ramesh1991AnProblem,Souffriau2008AGuides}. 

In literature, a number of meta-heuristics have been commonly employed to solve different \gls{op} variants. Swarm algorithms such as \Gls{aoc} \parencite{Ke2008AntsProblem, Wagner2016StealingProblem, Mukhina2019OrienteeringConstruction} and \gls{pso} \parencite{Sevkli2010StPSO:Optimization, Wagner2016StealingProblem} are meta-heuristic that have commonly found in literature. Vansteenwegen et al.\parencite{Vansteenwegen2011TheSurvey}  in their survey listed \gls{aoc} based algorithms as one of the best performing algorithms for the \gls{top} in their experiments. Memetic algorithms as used in \parencite{Lu2018AConstraints,Bouly2010AProblem, Divsalar2014ASelection} have also been widely employed in solving \gls{op} problems. Evolutionary algorithms that mimic genetics are also an algorithmic trend in literature\parencite{Wu2018EvolutionaryProblem, Faeda2020AProblem, Kobeaga2018AnProblem}. Swarm algorithms, evolutionary algorithms, memetic algorithms and simulated annealing as used in \parencite{Pan2018IndependentTourist}, are meta-heuristics that are commonly combined with standard heuristics such as tabu search\parencite{Lu2018AConstraints} or local search\parencite{Divsalar2014ASelection,Bouly2010AProblem} to achieve best results. For example, Labadie et al. developed a hybridized evolutionary local search algorithm for \gls{toptw} that showed an improvement against 150 best known solutions as shown in \parencite{OP_Solution_Gunawan}. Exploring the strengths of these state-of-the-art meta-heuristics are what recent studies have focused on.  However, standard heuristics such as the Branch-and-cut algorithm used in \parencite{Dang2013AProblem} and tabu search used in \parencite{E.2005AProblem} are known to show great results for the \gls{top}. 

\section{Related Work}
From research, none of the \gls{op} variants as commonly found in literature models our \gls{ttdp} one-to-one. Sivaramakrishnan et al.\parencite{MCKP_CustomProducts} used a \gls{mckp} to model a knowledge-based recommender system for customizable products. They suggested sequence of diverse product features to users based on preferred features without prior knowledge about the user. In such a domain, there is a cost threshold and the features suggested in a sequence have to be heterogeneous. The non-similarity in features modelled as classes into \gls{mckp} is related to the diversity in items we aim to achieve, thus their ideas can be applied to our domain. 

Burg et al.\parencite{Oregon_Trail_Knapsack} in their work defined an Oregon trail knapsack problem. Their idea was inspired by the Oregon trail game where players are asked to imagine preparing for a trek across the Oregon Trail. With a given amount of money to spend, and supplies weight limit, the travelers need to get good value for the supplies they purchase in order to make it across country. The aforementioned knapsack problem extension can be considered similar to our knapsack problem, however, the Oregon trail knapsack problem does not add time constraint to travel node as found in standard \gls{op}. Wörndl et al.\parencite{cbrecsys2014} have previously conducted a study on combining multiple travel regions into a composite trip. Their study is closely related to the subject matter of this thesis. The knapsack problem was formulated as an Oregon trail knapsack, however, intrinsic details about the dynamic programming approach used in the study is unclear. The study was proven to satisfy expert users more than baseline algorithms. Nonetheless, high diversity in regions was not achieved by their algorithm\parencite{cbrecsys2014}.

This work seeks to improve existing work by drawing on existing theories and methods as found in literature. A robust framework of problem model and scalable algorithmic approaches to build our recommender system for recommending diverse,  sequence of travel regions based on given user preference, shall be developed in subsequent sections.
