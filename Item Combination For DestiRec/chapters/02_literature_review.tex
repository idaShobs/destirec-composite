%The literature review is a core element of your thesis and shows that you are capable of working scientifically. As you explain what other researchers have found on your topic, the reader will realize that you know this topic extremely well. This will build trust that you can provide a piece of work yourself that is scientifically relevant. Equally important, you will need to identify a gap in the literature that you intent to fill. This is how you justify your thesis, and it helps the reader to assess the importance of your work. This gap may be methodological ("I will develop a new method that is able to answer my research question, which previously applied methods cannot as well."), use new data ("Other researchers used database X, but I will use data retrieved by Y."), or a new application ("This method has never been applied to the city of Munich.").
\chapter{Literature Review}\label{chapter:literature_review}