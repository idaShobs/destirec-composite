%The literature review is a core element of your thesis and shows that you are capable of working scientifically. As you explain what other researchers have found on your topic, the reader will realize that you know this topic extremely well. This will build trust that you can provide a piece of work yourself that is scientifically relevant. Equally important, you will need to identify a gap in the literature that you intent to fill. This is how you justify your thesis, and it helps the reader to assess the importance of your work. This gap may be methodological ("I will develop a new method that is able to answer my research question, which previously applied methods cannot as well."), use new data ("Other researchers used database X, but I will use data retrieved by Y."), or a new application ("This method has never been applied to the city of Munich.").
%Summarize and synthesize: give an overview of the main points of each source and combine them into a coherent whole
%Analyze and interpret: don’t just paraphrase other researchers—add your own interpretations where possible, discussing the significance of findings in relation to the literature as a whole
%Critically evaluate: mention the strengths and weaknesses of your sources
%Write in well-structured paragraphs: use transition words and topic sentences to draw connections, comparisons and contrasts

\chapter{Literature Review}\label{chapter:literature_review}
In 2007, Vansteenwegen and Oudheusden \parencite{Vansteenwegen2007TheOpportunity} introduced a popular \gls{ttdp} design. Their paper defined features that can be described as a template for the next-generation \gls{mtg}. Based on their research, the authors projected that the next generation MTG would become more personalized with components such as user profiles, attraction information, and trip information playing essential roles. They also speculated that \gls{or} would play a pivotal role in conceiving and optimizing tourist trips. Thus, they defined the \gls{ttdp} as a decision tool.

With their groundbreaking projections and proposals, technologies such as \gls{ai}, \gls{iot} and \gls{cps}, have brought about a paradigm shift in the possibilities for a dynamic tourist destination guide. Many recent studies have focused on designing and optimizing tourist trips with the aid of technological advancements. \Gls{ttdp} as a decision tool for next-generation \gls{mtg} has been thoroughly researched. However, this research has recently become less mainstream because of options offered by newer technologies. It has become the de facto approach to personalizing new generation tourist guides and trip \glspl{rs}. We explore such personalized \glspl{rs} after surveying recommendation techniques commonly found in the literature.


\section{Recommendation Techniques}
Generally, \glspl{rs} can be distinguished based on the issues they focus on and the techniques they use. As shown in Figure \ref{fig:recSystems-approaches}, \glspl{rs} can be based on collaborative filtering (also known as social filtering), content filtering, and knowledge-based filtering. Systems designed according to collaborative filtering identify users whose preferences are similar to those of the given user by computing similarities between users and items they have liked \parencite{Balabanovic1997Content-BasedRecommendation}. \Glspl{rs} use this filtering method extensively. Collaborative filtering often involves model-based and memory-based approaches. Model-based approaches typically use machine learning or statistical methods to build models based on user records for future recommendations \parencite{Chu2012DoesImages}. In contrast, memory-based approaches compare user data with previously stored data of other users \parencite{Schiaffino2006PoliteAgents}. This technique typically suffers from a cold-start problem. Such a problem occurs when there is not enough information about an item based on another user either rating it or specifying which other items it is similar to \parencite{Balabanovic1997Content-BasedRecommendation}. Also, data becomes sparse when there are not enough users available to cover the needed collection of recommendable items. The authors of \parencite{Gao2015Content-awareNetworks, Lian2015Content-AwareData} analyzed a set of items liked by a user in the past and tried to recommend similar items, resulting in a judgment of relevance that represents users’ preferences. Thus, it is highly advantageous in its effectiveness as an information retrieval process. Systems based on content filtering suffer from over-specialization. This is a function of the similarity between the recommended items based on users’ preferences at different points \parencite{Lops2011Content-basedTrends}. It also suffers from a cold-start problem whereby the system needs to have adequate user historical data to generate quality results \parencite{Burke2002HybridInteraction}. 

Knowledge-based systems \parencite{Burke2000Knowledge-basedSystems} depend on knowledge models of the object domain for effective item recommendation. They are based on the needs and preferences the user provides. Constraint-based techniques \parencite{Choi2021ATourismcity} allow systems to use domain knowledge regarding the features that cannot be represented by the user’s record, or in the absence of the user’s record, for recommending items. Case-based methods \parencite{Montejo-Raez2011Otium:Leisure} typically utilize the experiences and expertise of travel agents. Knowledge-based methods do not suffer from the problems faced in collaborative and content-filtering systems; their recommendations do not depend on a database of user ratings, and the systems do not have to gather information about a particular user because its judgments are independent of individual tastes \parencite{Burke2000Knowledge-basedSystems}. Therefore, knowledge-based systems are perfect complements to other \gls{rs} types.

Hybrid approaches \parencite{Adomavicius2005TowardExtensions,Ghazanfar2010BuildingFiltering} combine two or more of the approaches mentioned above to mitigate the weaknesses of each approach and complement their strengths for optimal performance.


Depending on the recommendation technique employed, destination \glspl{rs} can be classified as content based, collaborative, or knowledge based. However, a good deal of recent research has heavily focused on model-based collaborative filtering.

\begin{figure}[htpb]
  \centering
   \documentclass{standalone}
\begin{document}
\tikzset{
  basic/.style  = {draw, text width=2cm, drop shadow, font=\sffamily, rectangle},
  root/.style   = {basic, rounded corners=2pt, thin, align=center, fill=TUMAccentBlue, inner sep=6pt},
  level 2/.style = {basic, rounded corners=6pt, thin,align=center, fill=TUMAccentBlue, text width=8em, align=center,  inner sep=6pt},
  level 3/.style = {basic, thin, align=left,  text width=6.5em, align=center,  inner sep=4pt}
}
 \begin{tikzpicture}[
    level 1/.style={sibling distance=5cm},
    level 2/.append style={sibling distance=35mm},
   level distance = 2cm,  edge from parent fork down]
    
    \node [root]{Recommender Systems}
        child {node [level 2]{Collaborative \\ filtering}
        child {node [level 3] {Model-Based}}
        child {node [level 3] {Memory-Based}}
        }
        child {node [level 2] {Content-Based \\ filtering}}
        child {node [level 2] [sibling distance =2cm] {Knowledge-Based\\ filtering}
        child {node [level 3] {Constraint-Based}}
        child {node [level 3] {Case- \\Based}}
        };
    \end{tikzpicture}
\end{document}
    \caption[Recommender Systems Approaches Overview]{An overview of approaches used in recommender systems.}\label{fig:recSystems-approaches}
\end{figure}

\section{POI Recommendation}
The proliferation of social media as a societal norm has paved the way for \Glspl{lbn} like Foursquare and Facebook. Consequently, a new study area has been established in destination recommendation research. \Gls{poi} recommendation, as it is commonly known, is an active research area that focuses solely on recommending destinations to users on \gls{lbn}. Such systems utilize readily available check-in records of a user on the network to recommend further possible \glspl{poi} to the user. Advancements in \gls{poi} recommendation need to be explored because they constitute a significant step toward extracting user preference data that can be generalized to improve destination recommendation in systems that are outside location-based networks.

The main objective of \gls{poi} recommendation on \glspl{lbn} is to optimize the user experience by recommending \glspl{poi} based on users’ historical records and sometimes on collaborative information. \Glspl{lbn} have been researched in recent years due to the spatial-temporal-social information embedded in them \parencite{Cheng2011ExploringServices}. Typically, users check in with timestamps, location details, and their preferences that inform the \gls{poi} ratings, which can be accessed via an application program interface provided by the \gls{lbn} provider. As a consequence, many recent studies have experimented with representing the data sets through statistical or machine models to mine patterns in the \gls{lbn} data for the \gls{poi} prediction task. The data characteristics allow the definition of new properties that can be mined from them.

\Gls{lda} is a natural language processing statistical model that has been heavily adopted by researchers to extract \gls{poi} topics from comment sections \parencite{Liao2018POIFactorization, Huang2020Multi-modalNetworks}. Liao et al. \parencite{Liao2018POIFactorization} explored comment relations using \gls{lda}. In their study, their latent features were user-topic-time \glspl{tensor}. The \glspl{tensor} were constructed by connecting check-in data with a \gls{poi} topic generated by a \gls{lda} model. The \gls{poi} model extracts the topics based on the comments given by users, the topics, and \gls{poi}-topic distributions of all \glspl{poi}.

Some of the most successful realizations of statistical models for \glspl{rs} are based on matrix factorization \parencite{Cheng2013WhereRecommendation, Cheng2012FusedNetworks, Chen2018PrivacyFactorization, Gao2013ExploringNetworks, Lian2015Content-AwareData}. Matrix factorization \parencite{Koren2009MatrixSystems} algorithms work by mapping items to a joint latent factor space of dimensionality such that user-item interactions are modeled as inner products in that space. Consequently, high correspondence between item and user factors leads to a recommendation. A major strength of matrix factorization is that it allows the incorporation of additional information. In a \gls{lbn}, the social relations between users help in incorporating similarity information between users’ interests into matrix factorization \parencite{Cheng2013WhereRecommendation, Liao2018POIFactorization, Huang2020Multi-modalNetworks}. Liao et al.\parencite{Liao2018POIFactorization} used a \gls{hosvd} of third-order \glspl{tensor} to recommend \glspl{poi}. \Gls{hosvd} can be considered a generalization of a matrix singular value decomposition \parencite{Vandewalle1990SingularProcessing}. However, factorizing the matrix often raises difficulties due to the high portion of missing values caused by sparseness in the matrix resulting from cold start. Huang et al. \parencite{Huang2020Multi-modalNetworks} addressed this by simultaneously mining the sequential, temporal, and spatial patterns of users’ check-in behavior such that data were abundant. Others including Cheng et al.\parencite{Cheng2013WhereRecommendation} employed a low-rank approximation of sparse, partially observed \glspl{tensor}.

In addition to the social relations that can be incorporated into matrix factorization or topics, many existing studies have used the Markov-chain property of inter check-in behavior to model the sequential check-in pattern of users \parencite{Cheng2013WhereRecommendation, Zhao2016STELLAR:Recommendation, Yang2013ASystem}. Cheng et al. \parencite{Cheng2013WhereRecommendation} explored the geographical relation of \gls{lbn} inter check-in distance in order to recommend successive \glspl{poi} to users by using a personalized Markov chain (FPMC). Liu et al. \parencite{Liu2016PredictingContexts} employed recurrent neural networks (RNN) to detect subsequent correlations of check-in sequences.

Generally, due to the abundance of \gls{lbn} data available to these models, the results obtained in many studies of \gls{poi} recommendation are mostly improvements to previous studies. Additionally, applying model-based techniques and using approximation algorithms has helped achieve a breakthrough regarding the cold-start problem that sometimes occurs in collaborative filtering. These steady improvements in \gls{poi} recommendation precision and accuracy are indeed groundbreaking and constitute a significant step toward improving destination recommendation as a whole.


\section{OP solving methodologies}
Though \gls{poi} recommendation is an attractive research area, much research has also been done into destination recommendation outside \glspl{lbn}. The \gls{op} that originates from operations research is still regarded as a tool for modeling the \gls{ttdp} and, by extension, solving the problem of destination recommendation. According to \parencite{Vansteenwegen2011TheSurvey}, given a set of nodes $N = \{1,...,|N|\}$ where each node $i \in N$ is associated with a non-negative score $S_i$, and the nodes $1$ and $N$ are the start and end nodes, respectively, the goal of the \gls{op} is to determine a path, limited by a given time budget $T_{max}$, that visits a subset of $N$ and maximizes the total collected score. Each collected score can be added, and each node can be visited at most once. Many researchers have solved different problems by formulating them as variants of the \gls{op}; these variants include the \gls{op}. Such variants include \gls{top}, \gls{tdop}, \gls{optw}, \gls{toptw}, \gls{opsp}, \gls{moop}, and more recently the \gls{thop}.

Gunawan et al. (2016) \parencite{OP_Solution_Gunawan} published a paper surveying all recent variations, solution approaches, and applications of \gls{op} variants. Although numerous studies have thoroughly researched some \gls{op} variants, it is nearly impossible to find studies for some other variants in the literature. \Gls{moop} is the variant closest to our objective; it derives its name from multi-objective optimization in OR in which there is more than one objective function. A \gls{moop} assigns a set of scores $S_{ik}$ to each \gls{poi} to model the different scores a \gls{poi} might have for each attraction category $k \in \{1,...,m\}$. Exact algorithms for the \gls{op} are complex and computationally time-consuming because it is NP-hard \parencite{Golden1987TheProblem}. Therefore, heuristic approaches such as those found in \parencite{T.1984HeuristicOrienteering, Golden1987TheProblem, Ramesh1991AnProblem,Souffriau2008AGuides} are the focus of most research. 

A good deal of literature uses several meta-heuristics to solve different \gls{op} variants. Most of the meta-heuristics mimic natural metaphors to solve optimization problems (e.g., evolution of species, annealing process, ant colony, particle swarm, immune system, bee colony, and wasp swarm). Swarm algorithms such as \Gls{aoc}  \parencite{Ke2008AntsProblem, Wagner2016StealingProblem, Mukhina2019OrienteeringConstruction, Martin-Moreno2018Multi-ObjectiveProblem} and \gls{pso} \parencite{Sevkli2010StPSO:Optimization, Wagner2016StealingProblem, Schilde2009MetaheuristicsProblem} are meta-heuristics commonly found in literature for solving two-objective and multi-objective \glspl{op}. In their survey Vansteenwegen and Oudheusden \parencite{Vansteenwegen2011TheSurvey} found \gls{aoc}-based algorithms to be the best-performing algorithms for the TOP in their experiments. Memetic algorithms as used in \parencite{Lu2018AConstraints,Bouly2010AProblem, Divsalar2014ASelection} have also been used to solve different variants of \gls{op} problems. \Glspl{ea} that mimic genetics constitute an algorithmic trend in the literature \parencite{Wu2018EvolutionaryProblem, Faeda2020AProblem, Kobeaga2018AnProblem}. Swarm algorithms, \gls{ea}, memetic algorithms, and simulated annealing as used in \parencite{Pan2018IndependentTourist}, are commonly combined with other heuristics such as tabu search \parencite{Lu2018AConstraints} or pure local search \parencite{Divsalar2014ASelection, Bouly2010AProblem} to achieve the best results. For example, Labadie et al. \parencite{Labadie2011HybridizedWindows} developed a hybridized evolutionary local search algorithm for \gls{toptw} that showed an improvement over the 150 best-known solutions, as shown in \parencite{OP_Solution_Gunawan}. Recent studies have focused on exploring the strengths of many of these state-of-the-art meta-heuristics. However, standard heuristics such as the branch-and-cut algorithm used in \parencite{Dang2013AProblem} and tabu search used in  \parencite{E.2005AProblem} are known to show positive results for the \gls{top}. 

\section{Related Work}
A good deal of literature focuses either on travel recommendation or bundling items in a recommendation package. However, there is little research into bundling items for destination recommendation. A non-time-dependent \gls{moop} best describes our \gls{op} variant. To the best of our knowledge, no researchers have carried out travel recommendations using this \gls{op} variant. Martín-Moreno and Vega-Rodríguez \parencite{Martin-Moreno2018Multi-ObjectiveProblem} proposed a swarm intelligence-based algorithm to solve a time-dependent \gls{moop}. However, due to little work having been done in \gls{op} for multi-objectives, benchmark instances to compare the performance of their algorithm are two-objective \gls{op} based. Moreover, the \gls{moop} presented seeks to find optimal tours and not paths (i.e., start at node A and end at node A).

At the core of this thesis are aspects relating to other works found in the literature. The following sections highlight the two most closely related works.


\subsection{The Oregon Trail Knapsack} \label{sec:oregon}
Burg et al. \parencite{Oregon_Trail_Knapsack} defined an Oregon Trail knapsack problem that was inspired by Oregon Trail game. The game asks players to imagine preparing for a trek along the Oregon Trail. With a given amount of money to spend and a weight limit on supplies, the travelers need to get good value for the supplies they purchase to make it across the country. This extension of the knapsack problem can be considered similar to our knapsack problem. However, the Oregon Trail knapsack problem does not add a time constraint to the travel node as found in a standard \gls{op}. The authors’ formal definition of the problem is as follows:
%\intertext{subject to}
\begin{align}\tag{1}
    maximize \qquad &\sum_{j=1}^n f_j(x_j, x_{d_j})\label{eq:2a}\\
    \tag{2}subject \hspace{0.1cm} to \qquad &\sum_{j=1}^n w_j x_j \leq W\label{eq:2b} \\
  \tag{3} &\sum_{j=1}^n c_j x_j \leq C\label{eq:2c} \\
   \tag{4}  0\leq x_j \leq b_j, \qquad &\forall \hspace{0.1cm} 1 \leq j \leq n\label{eq:2d}
\end{align}
where $W$ in Equation \ref{eq:2b} is the weight limit, and $C$ in Equation \ref{eq:2c} is the cost limit. The function in Equation \ref{eq:2a} is the value function for Type $j$, in which $d_j$ gives the index of the type upon which the value of $x_j$ depends.  $d_j \in \{d_1, d_2,...,d_n\}$ provides the dependency information among the item types. The value function in this model is of particular interest because it models three possible value types of the items. Burg and Lang\parencite{Oregon_Trail_Knapsack} define the value functions as follows:

\noindent
\begin{flalign}
\tag{type 1 }\qquad f_j(x_j, x_{d_j}) &= x_j \cdot v_j \cdot [x_{d_j} > 0] \label{eq:2e}&&\\
\tag{type 2} \qquad f_j(x_j, x_{d_j}) &= [x_{d_j} > 0] \cdot \sum_{i=0}^{x_j - 1 } r^i \cdot v_j \label{eq:2f}&&\\&= [x_{d_j} > 0] \cdot v_j \cdot \frac{1-r^{x_j}}{1 - r}, \hspace{2.1cm} \hfill\forall \hspace{0.1cm} 0 < r < 1\notag &&\\
\tag{type 3} \qquad f_j(x_j, x_{d_j}) &= x_j \cdot v_j - [x_{d_j} > 0] \cdot t \cdot x_j \cdot v_j, \qquad  \hfill \forall \hspace{0.1cm} 0 < t \leq 1  \wedge d_j \neq j\label{eq:2g}
\end{flalign}

Each function has a value constant given by $v_j$. The Iverson notation $[x_{d_j} > 0]$ denotes the boolean value $x_{d_j} \in \{0,1\}$, which is when  $x_{d_j} > 0 $. Value function Type \ref{eq:2e} represents the case, in which items of Type $j$ only have a value if at least one item of Type $d_j$ is present in the solution. The value function Type \ref{eq:2f} covers the value type, in which with each added item of Type $j$, the item value $v_j$ diminish at a rate of $r$. Once more, the function adds the value of items of Type $j$ if at least one item of type $d_j$ is in the knapsack. Lastly, the Type-\ref{eq:2g} function represents the case where a factor of $t$ reduces the value associated with $x_j$ of Type-$j$ items, when Type-$x_{d_j}$ items are in the solution. 

\subsection{Composite Trip Recommendation}
Wörndl and Herzog \parencite{cbrecsys2014} researched combining multiple travel regions into a composite trip. Their algorithmic approach comprises three phases:
\begin{enumerate}
    \item Reduce number of regions
    \item Rate regions
    \item Calculate the best combination of regions
\end{enumerate}

Phases 1 and 2 are preliminary steps taken before the actual recommendation process. This approach allows users to exempt regions from their query. The hierarchical tree data model used enables the procedure to exempt sub-regions of exempted regions. Thus, the number of possible regions to explore is significantly reduced before starting the next phases. In Phase 2, an asymmetric similarity metric is used to assign preference ranks to regions. The asymmetrical similarity metric allows region features to be assigned weights. For example, the traveling period is weighted higher than features such as crime level. Regions with low scores are then removed from consideration before the next step. Finally, the best combinations of regions are calculated using dynamic programming. Intrinsic details regarding the dynamic programming approach used in the developing their travel package algorithm are unclear. The algorithm was shown to satisfy expert users more than a baseline knapsack algorithm and a top-k algorithm.


\section{Conclusion}
In this chapter, we reviewed different techniques found in literature for item recommendation. We investigated state-of-art methodologies found in the literature for general \gls{poi} recommendation and then went on to review how \glspl{op} have been solved in the literature. Wörndl and Herzog \parencite{cbrecsys2014} did substantial work on item combination for destination \glspl{rs}. However, their algorithm fails to achieve high diversity in regions \parencite{cbrecsys2014} and their approach fails to scale. Penalty-based approaches as used in their work can either improve the heuristic’s efficiency, or it can restrict the effectiveness of the heuristic. This depends on the penalty function definition and the chosen penalty value. Our work seeks to improve the work of Wörndl and Herzog by exploring methodologies and techniques to efficiently optimize the item-combination procedure. In addition, our thesis contributes to existing research on \glspl{op} by characterizing a multi-objective, non-time-dependent \gls{op}. Thereby, more efficient heuristics for computing combinations of regions can be explored. Overall, higher diversity in recommended regions is ensured. A robust framework of problem-model and scalable algorithmic approaches to building our \gls{rs} for recommending diverse sequences of travel regions based on given user preferences is developed in subsequent chapters.
