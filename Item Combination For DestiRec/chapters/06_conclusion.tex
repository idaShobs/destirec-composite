%The most important at the beginning: The chapter Conclusions does not contain a summary of your thesis! The summary is provided in the abstract of the beginning of your thesis, but not here. Instead, the conclusions shall do what the title suggests: Synthesize your findings and conclude what we learn from that. It will be useful to refer to your research question(s) and discuss if those were confirmed or rejected by your research. You may also refer back to you literature review and compare your findings with the findings that others have published. This is also a good place to talk about limitations of your research. By clearly stating what your research is not able to do well, your thesis becomes stronger. If you show that you understand what your methodology misses, you show the reader that you understand very well what you research has accomplished, and what may need further research. Which brings us to another topic you should touch on in your conclusions: What are future research needs? If a fellow student of you wanted to build on your research, what would be the next logical step that (s)he should try to address? Last but not least, you may also assess if your findings have practical implications. Examples: Should waste water engineers use an additional test to assess water quality? Should transportation planners use different data to assess the level of service?
\chapter{Conclusion}\label{chapter:conclusion}

\section{Limitations}

\section{Future Research}