
\chapter{Conclusion}\label{chapter:conclusion}
In this thesis, we formally introduced the problem of item combination in destination \glspl{rs} as a \gls{moop}. Empirical research into how current state-of-the-art algorithms used for solving \glspl{op} and general \glspl{moop} can be extended to efficiently combine items in destination \glspl{rs} was performed. A comparison of choice algorithms was made, and it was decided that an evolutionary algorithm is best-suited for our problem class. The \gls{nsga}-III, a dominance-based genetic algorithm, was chosen to obtain a sequence of candidate solutions that satisfy user constraints.


Six heuristic variations of the algorithm were implemented. Evaluations conducted using an online study, and through a user survey showed that initializing the algorithm with feasible population members only and penalizing the individuals based on their distance to a feasible area had the highest rating of all variants on all defined metrics. Pure randomness in the initialization of population members showed poor results. However, incorporating composite penalization of the individuals generated by random initialization improved performance significantly.


Similarity-based initialization, as used in \parencite{cbrecsys2014} and further investigated in this work, produced poor results. This initialization technique restricts the search space such that there are a limited number of regions to combine during the actual evolutionary process. Combining items for recommendation is heavily dependent on the defined constraints. Similarity-based initialization does not take the constraints into account. Hence, during the evolutionary step in which the actual constraints determine the fitness of an individual, the limited regions in the population perform poorly. This also explains the poor performance in terms of diversity shown by the travel package algorithm developed by Wörndl and Herzog; the algorithm also used an equivalent similarity metric.

In general, this thesis has shown the potential applicability of \glspl{ea} in efficiently combining items in destination \glspl{rs}.


\section{Limitations}
The results obtained on metrics such as accuracy, satisfaction, cohesiveness, and diversity by this work are above average but imply possible improvement points through further research. Higher ratings from users in terms of accuracy and satisfaction are needed. The computational time of the algorithm is crucial for the usability of a \gls{rs}. Evaluations have shown that a considerable amount of time is needed to complete the evolutionary steps. Therefore, the current implementations of this thesis are not suitable for synchronous \gls{rs}. However, the algorithm could be implemented into an asynchronous \gls{rs}. The underlying data for this research influenced the limitations on recommended travel budgets for a region. Some regions in the database had no data regarding the recommended budget. Consequently, we excluded the regions' budget recommendations from suggested region combinations.



\section{Future Research}
Research, implementation, and evaluations from this thesis can be considered as a baseline for future work on item combination for destination \glspl{rs}. We investigated the applicability of genetic \glspl{ea} to this topic, and we introduced possibly applicable algorithms, for example \gls{aoc}. A possible direction for future work could be to investigate this algorithm type and compare its results with the results obtained in this work.

Further research into \glspl{ea} for item combination in \glspl{rs} has to be done. With this work as a baseline, investigating the trade-offs between the number of objectives, maximum generation, and population size could be a possible research focus. Finding the right balance between the three parameters is vital to improving computational speed while ensuring optimal item combinations.

Additionally, results from user survey proved that travelers with advanced knowledge are significantly different from travelers with lesser knowledge about destinations. This suggests a need for further user surveys with travel experts. Wörndl and Herzog had initially conducted their survey with travel experts only. Their survey method is indeed a better approach for evaluating results from an implemented algorithm.

An improved database of regions with the needed budgets and updated scores for different activities is vital and constitutes possible future work. A transformation and enhancement of the region names with ISO 3166 transformation would significantly increase the real-world usage of the database. For example, it would be possible to obtain the latitude and longitude coordinates of the regions if their ISO codes were added to the database. This would improve accuracy in routing and the cohesiveness of recommended regions.