\chapter{\abstractname}
Recommender systems produce individualized recommendations by aggregating users preference. The issue of item combination in travel recommender systems is a computationally hard problem. Numerous works have studied point-of-interest recommendation for travel recommender systems as a time dependent orienteering problem. However, limited work have studied the topic of item combination as a non-time dependent orienteering problem

In this thesis, we establish multi-objective evolutionary algorithms as a novel approach to solving the non-time-dependent orienteering problem. Extensive study of various state-of-the-art algorithms used to solve the orienteering problem is conducted. Then, a mathematical reformulation of the problem as a multi-objective orienteering problem is presented. Further empirical research of meta-heuristic solutions to the problem definition resulted in selecting a non-dominated genetic evolutionary algorithm for solving the problem. 

The NSGA-III and five heuristic variations thereof were tested and evaluated. An evaluation of the experiments was conducted online based on computational speed and high fitness score. An offline user evaluation was also conducted. It was discovered that a beginning population consisting only of feasible individuals and penalty-based constraint handling using individuals’ distances to the feasible region was rated best by users regarding accuracy, satisfaction, cohesiveness, and diversity.


\makeatletter
\ifthenelse{\pdf@strcmp{\languagename}{english}=0}
{\renewcommand{\abstractname}{Kurzfassung}}
{\renewcommand{\abstractname}{Abstract}}
\makeatother

\chapter{\abstractname}
\begin{otherlanguage}{ngerman}
Empfehlungssysteme erstellen individualisierte Empfehlungen auf der Grundlage zusammengefasster Präferenzen der Nutzer. Das Problem der Item-Kombination in Reiseempfehlungssystemen ist ein rechenintensives Problem. Zahlreiche Arbeiten haben Point-of-Interest-Empfehlungen für Reiseempfehlungssysteme als ein zeitabhängiges Orientierungsproblem untersucht. Es gibt jedoch nur wenige Arbeiten, die sich mit dem Thema der Item-Kombination als nicht-zeitabhängiges Orientierungsproblem beschäftigt haben.

In dieser Arbeit definieren wir das Problem der Item-Kombination als ein nicht-zeitabhängiges Orientierungsproblem und etablieren multi-objektive evolutionäre Algorithmen als einen neuen Ansatz zur Lösung des Problems. Es wird eine umfassende Studie verschiedener moderner Algorithmen durchgeführt, die zur Lösung des Orientierungsproblems verwendet werden. Anschließend wird eine mathematische Neuformulierung des Problems als multikriterielles Orientierungsproblem vorgestellt. Weitere empirische Untersuchungen von meta-heuristischen Lösungen für die Problemstellung führten zur Auswahl eines nicht-dominanten genetischen Evolutionsalgorithmus für die Lösung des Problems.

Der NSGA-III und fünf heuristische Varianten davon wurden experimentell untersucht und bewertet. Die Bewertung der Experimente erfolgte online anhand der Rechengeschwindigkeit und der hohen Fitnesswerte. Außerdem wurde eine Offline-Evaluierung mittels Benutzerbewertung durchgeführt. Es wurde festgestellt, dass eine Startpopulation, die nur aus realisierbaren Individuen besteht, und eine penalty-basierte Beschränkungsbehandlung, die den Abstand der Individuen zur realisierbaren Region verwendet, von den Nutzern hinsichtlich Genauigkeit, Zufriedenheit, Kohäsion und Diversität am besten bewertet wurde.  

\end{otherlanguage}


% Undo the name switch
\makeatletter
\ifthenelse{\pdf@strcmp{\languagename}{english}=0}
{\renewcommand{\abstractname}{Abstract}}
{\renewcommand{\abstractname}{Kurzfassung}}
\makeatother